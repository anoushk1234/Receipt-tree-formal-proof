
\newcommand{\Deposit}{\ms{Deposit}}
\newcommand{\Withdraw}{\ms{Withdraw}}
\newcommand{\decimals}{\ms{decimals}}
\newcommand{\amount}{\ms{amount}}
\newcommand{\crypto}{\ms{zk}\text{-}\ms{proof}}
\newcommand{\pirange}{\pi_{\ms{range}}}

\newcommand{\dec}{\ms{decimals}}
\newcommand{\finalbalance}{\ms{final}\text{-}\ms{balance}}
\newcommand{\currpending}{\ms{curr}\text{-}\ms{pending}}

\newcommand{\writable}{\ms{write}}
\newcommand{\signer}{\ms{sign}}
\newcommand{\none}{\ms{none}}

\newcommand{\CreateAccount}{\ms{CreateAccount}}
\newcommand{\UpdateAccountPK}{\ms{UpdateAccountPK}}
\newcommand{\DisableInboundTransfers}{\ms{DisableInboundTransfers}}
\newcommand{\EnableInboundTransfers}{\ms{EnableInboundTransfers}}

\newcommand{\pizero}{\pi_{\ms{zero}\text{-}\ms{balance}}}
\newcommand{\piequality}{\pi_{\ms{equality}}}
\newcommand{\pivalidity}{\pi_{\ms{validity}}}

\newcommand{\current}{\ms{current}}
\newcommand{\new}{\ms{new}}

\newcommand{\TransferWithRangeProof}{\ms{TransferWithRangeProof}}
\newcommand{\TransferWithValidityProof}{\ms{TransferWithValidityProof}}

\newcommand{\ConfigureMint}{\ms{ConfigureMint}}

\clearpage

\section{Instructions}
\label{sec:instructions}

\red{
  The specifications of the instructions $\ConfigureMint$,
  $\UpdateTransferAuditor$, and $\Transfer$ will be included after these
  instructions become more stabilized. This section should include some
  discussion of the pre-instruction and other high-level implementation details.
}

\subsection{Managing ZK-Token Mints}



\todo{Discuss the next two instructions}


\begingroup
\setlength{\parindent}{0em}
\setlength{\parskip}{0.75em}


\subsubsection{$\ConfigureMint$}
To be written...

\subsubsection{$\UpdateTransferAuditor$}
To be written...

\endgroup

\newpage
\subsection{Managing ZK-Token Accounts}
Users can manage \texttt{ZK-Token} accounts via a number of instructions. Among
these, the $\CloseAccount$, $\UpdateAccountPK$, and $\Withdraw$
instructions require zero-knowledge proofs that certify the validity of the
instructions and state data.

Each \texttt{ZK-Token} account is uniquely associated with a single
\texttt{Token} account. These \texttt{ZK-Token} accounts can be created or closed
using the $\CreateAccount$ and $\CloseAccount$ instructions.
\begin{itemize}
  \item The $\CreateAccount$ instruction creates a \texttt{ZK-Token} account at the 
    address derived as a PDA from the linked \texttt{Token} account
    and its associated \texttt{Token} mint. The account's encryption key is set
    to be the ElGamal public key that is specified as part of the
    instruction data. The pending and available balance is set to be default
    encryptions of zero.

  \item The $\CloseAccount$ instruction closes a \texttt{ZK-Token} account,
    transferring any Lamports to a destination account that is specified in the
    instruction. It is required that the account to be closed does not hold any
    funds. The account's pending balance must be empty (a deterministic
    encryption of 0). The account's available balance must be an encryption of 0
    and a proper zero-balance proof (Section~\ref{sec:zero-balance-proof}) must
    be provided with the instruction.
\end{itemize}

\noindent
Funds can be transferred between regular \texttt{Token} accounts and its
associated \texttt{ZK-Token} accounts via the $\Deposit$ and $\Withdraw$
instructions.
\begin{itemize}
  \item The $\Deposit$ instruction transfers funds from a \texttt{Token} account
    to its associated \texttt{ZK-Token} account. 
    \begin{itemize}
      \item On the \texttt{Token} side, funds are transferred from
        the source \texttt{Token} account to an omnibus \texttt{Token} account
        associated with the token mint.
      \item On the \texttt{ZK-Token} side, the pending balance is
        (homomorphically) incremented by the deposit amount.
    \end{itemize}

  \item The $\Withdraw$ instruction transfers funds from a \texttt{ZK-Token}
    account to its associated \texttt{Token} account. It works in reverse to the
    $\Deposit$ instruction: funds are transferred from the omnibus
    \texttt{Token} account to a user's \texttt{Token} account, and the encrypted
    available balance in the \texttt{ZK-Token} account is decremented by the
    withdraw amount. As funds are removed from the \texttt{ZK-Token} account, a
    transaction must contain a proper range proof (Section~\ref{sec:rangeproof})
    that certifies that there are enough funds in the \texttt{ZK-Token} account.
\end{itemize}

\noindent
In the \texttt{ZK-Token} program, an account owner can update the
encryption key associated with an account by the $\UpdateAccountPK$ instruction.

\begin{itemize}
  \item The $\UpdateAccountPK$ updates the ElGamal public key in a
    \texttt{ZK-Token} account. If the key is updated in an account, the current
    pending and available balance that is encrypted under the older key become
    invalid. Therefore, to update the key in an account, it is required that:
    \begin{itemize}
      \item The pending balance is empty (deterministic encryption of 0).
        The instruction can, for instance, be preceded by an
        $\ApplyPendingBalance$ instruction.

      \item The $\UpdateAccountPK$ must include a spendable balance that is
        encrypted under the new public key and an equality proof
        (Section~\ref{sec:equality-proof}) that certifies the validity of the
        newly encrypted spendable balance. The \texttt{ZK-Token} program verifies
        the proof and then replaces the encrypted spendable balance with the
        new ciphertext (in addition to updating the key).

    \end{itemize}
\end{itemize}

\noindent
As discussed in Section~\ref{sec:overview-design}, instructions that require
zero-knowledge proofs may be sucsceptible to front-running attacks. As
zero-knowledge proofs are verified with respect to an account state, they can be
invalidated if the account state changes before the proofs are actually
processed. The funds in \texttt{ZK-Token} accounts are divided into
\emph{pending} balance and \emph{available} balance such that $\Withdraw$ and
$\Transfer$, the instructions that are projected to be used frequently, are not
susceptible to these type of attacks. Funds that are sent to a \texttt{ZK-Token}
account are always aggregated into its pending balance, and funds that are sent
from a \texttt{ZK-Token} account are always subtracted from its available
balance. An account's pending balance can be added to its available balance via
the $\ApplyPendingBalance$ instruction.
\begin{itemize}
  \item The $\ApplyPendingBalance$ instruction aggregates a \texttt{ZK-Token}
    account's pending balance into its avaialble balance. After the instruction
    is executed, the account's available balance becomes the sum of its pending
    and available balance. The account's the pending balance is set to be empty
    (a deterministic encryption of 0). 
\end{itemize}

\noindent
In contrast to the $\Withdraw$ and $\Transfer$ instructions, the $\CloseAccount$
and $\UpdateAccountPK$ instructions are still susceptible to front-running. The
\texttt{ZK-Token} program provide instructions to defend aginst potential
attacks.
\begin{itemize}
  \item The $\DisableInboundTransfers$ instruction disables an account from
    accepting any incoming transfers.

  \item The $\EnableInboundTransfers$ instruction enables an account to
    accepting incoming transfers.
\end{itemize}
If any $\CloseAccount$ and $\UpdateAccountPK$ instructions start failing, then
any inbound transfer to the account should be disabled first before submitting any
of these instructions.


\begingroup
\setlength{\parindent}{0em}
\setlength{\parskip}{0.75em}

\newpage
\subsubsection{$\CreateAccount$}
Create a \texttt{ZK-Token} account.

The account address is a PDA, derived from the \texttt{Token} mint and linked
\texttt{Token} account. Ownership is held in the linked \texttt{Token} account.
The new account is rent-exempt.

The instruction fails if the \texttt{ZK-Token} account already exists.

\vspace{0.5em}

\textit{Addresses to be specified}:
\begin{itemize}[leftmargin=2.5cm]
  \item[$\writable$, $\signer$] The funding account for rent (must be a system
    account)
  \item[$\writable$] The new \texttt{ZK-Token} account to create
  \item[] Corresponding \texttt{Token} account
  \item[] System program
  \item[$\signer$/$\none$] The single account owner or the multisig account owner
  \item[$\signer$, $\opt$]  Required $M$ signer accounts for the \texttt{Token}
    Multisig account
\end{itemize}

\noindent
\textit{Data to be specified}:
\begin{itemize}
  \item $\pk$: The ElGamal public key to be associated with the new account.
\end{itemize}

\noindent
\textit{Cryptographic checks}:
\begin{itemize}
  \item[] (none)
\end{itemize}


\newpage
\subsubsection{$\CloseAccount$}
Close a \texttt{ZK-Token} account by transferring all lamports it holds to the
destination account.

The account must not hold any tokens in its pending or available
balances. Use $\DisableInboundTransfers$ to block inbound transfers first if
necessary.

\vspace{0.5em}

\textit{Addresses to be specified}:
\begin{itemize}[leftmargin=2.5cm]
  \item[$\writable$] The \texttt{ZK-Token} account to close
  \item[] Corresponding \texttt{Token} account
  \item[$\writable$] The destination account
  \item[] Instruction sysvar
  \item[$\signer$/$\none$] The single account owner or the multisig account owner
  \item[$\signer$, $\opt$]  Required $M$ signer accounts for the \texttt{Token}
    Multisig account
\end{itemize}

\noindent
\textit{Data to be specified}:
\begin{itemize}
  \item $\ct_\available$: The current encrypted available balance in the
    \texttt{ZK-Token} account to close.

    This component is included as part of the instruction so that the
    ciphertext-proof pair $(\ct_\available, \pizero)$ can be verified as
    a pre-instruction.

  \item $\pizero$: Zero-knowledge proof certifying that $\ct_\available$
    encrypts $0$.
\end{itemize}

\noindent
\textit{Cryptographic checks}:
\begin{itemize}
  \item \textit{Pre-instruction}: Zero-knowledge proof verification
    $(\ct_\available, \pizero)$
  \item \textit{In-program}: The available balance in the \texttt{ZK-Token} account
    to be closed is equal to $\ct_\available$ in the instruction data.
  \item \textit{In-program}: The pending balance $\ct_\pending$ in the
    \texttt{ZK-Token} account to be closed is a deterministic encryption of 0:
    \[ \ct_\pending = (0H + 0G, 0P) .\]
\end{itemize}


\newpage
\subsubsection{$\Deposit$}
Transfer funds from a \texttt{Token} account to the corresponding
\texttt{ZK-Token} account.
\vspace{0.5em}

\noindent
\textit{Addresses to be specified}:
\begin{itemize}[leftmargin=2.5cm]
  \item[$\writable$] The source \texttt{Token} account
  \item[$\writable$] The destination \texttt{ZK-Token} account
  \item[] The destination's corresponding \texttt{Token} account
  \item[$\writable$] The omnibus \texttt{Token} account for this token mint
  \item[] The token mint
  \item[] The \texttt{Token} program
  \item[$\signer$/$\none$] The single account owner or the multisig account owner
  \item[$\signer$, $\opt$]  Required $M$ signer accounts for the \texttt{Token}
    Multisig account
\end{itemize}

\noindent
\textit{Data to be specified}:
\begin{itemize}[leftmargin=1.25cm]
  \item $(\amount, \dec)$: The deposit amount and decimals
\end{itemize}

\noindent
\textit{Cryptographic checks}:
\begin{itemize}
  \item[] (none)
\end{itemize}

\newpage
\subsubsection{$\Withdraw$}
Transfer funds from a \texttt{ZK-Token} account to a regular \texttt{Token}
account.
\vspace{0.5em}

\noindent
\textit{Addresses to be specified}:
\begin{itemize}[leftmargin=2.5cm]
  \item[$\writable$] The source \texttt{ZK-Token} account
  \item[] The corresponding \texttt{Token} account to source
  \item[$\writable$] The destination \texttt{Token} account
  \item[] The token mint
  \item[$\writable$] The omnibus \texttt{Token} account for this token mint
  \item[] The \texttt{Token} program
  \item[] Instructions sysvar
  \item[$\signer$/$\none$] The single source account owner or the multisig
    source account owner
  \item[$\signer$, $\opt$]  Required $M$ signer accounts for the \texttt{Token} Multisig
    account
\end{itemize}

\noindent
\textit{Data to be specified}:
\begin{itemize}
  \item $(\amount, \dec)$: The withdraw amount and decimals

  \item $\ct_\available$: The expected encrypted available balance
    \emph{after} $\amount$ is withdrawn from the \texttt{Token} account.

    This component can be computed by the program itself; it is included
    as part of the instruction so that the ciphertext-proof pair
    $(\ct_\available, \pirange)$ can be verified as a pre-instruction.

  \item $\pirange$: Zero-knowledge proof certifying that $\ct_\available$
    encrypts a value $0 \le x < 2^{64}$.
\end{itemize}

\noindent
\textit{Cryptographic checks}:
\begin{itemize}
  \item \textit{Pre-instruction}: Zero-knowledge proof verification
    $(\ct_\available, \pirange)$
  \item \textit{In-program}: The algebraic relation to verify the
    well-formedness of $\ct_\available$:
    \[ \ct_\available = \ct'_\available - \ct_\amount \]
    where $\ct'_\available$ is the current encrypted available balance in
    the source \texttt{ZK-Token} account and $\ct_\amount = (x_\amount G, 0P)$
    is the (deterministic) encryption of $\amount$.
\end{itemize}


\clearpage

\subsubsection{$\UpdateAccountPK$}
Update the ElGamal public key in a \texttt{ZK-Token} account.

This instruction will fail if the pending balance is not empty and therefore,
invoking the $\ApplyPendingBalance$ instruction prior to this instruction is
recommended.

If the account is heavily used, consider invoking $\DisableInboundTransfers$ in
a separate transaction first to avoid new inbound transfers from causing this
instruction to fail.
\vspace{0.5em}

\noindent
\textit{Addresses to be specified}:
\begin{itemize}[leftmargin=2.5cm]
  \item[$\writable$] The \texttt{ZK-Token} account to update
  \item[] The corresponding \texttt{Token} account for the \texttt{ZK-Token}
    account
  \item[$\signer$/$\none$] The single account owner or the multisig account
    owner
  \item[$\signer$, $\opt$]  Required $M$ signer accounts for the \texttt{Token} Multisig
    account
\end{itemize}

\noindent
\textit{Data to be specified}:
\begin{itemize}
  \item $\pk_\new$: The new encryption key for the \texttt{ZK-Token} account.
    
  \item $\ct_\new$: The encrypted available balance under the new encryption
    key~$\pk_\new$.

  \item $\ct_\current$: The current encrypted available balance in the
    \texttt{ZK-Token} account to update.

    This component is included as part of the instruction so that the
    ciphertexts and proof $(\ct_\new, \ct_\current, \piequality)$ can be
    verified as a pre-instruction.

  \item $\piequality$: Zero-knowledge proof certifying that $\ct_\new$ and
    $\ct_\current$ encrypt the same value.

\end{itemize}


\noindent
\textit{Cryptographic checks}:
\begin{itemize}
  \item \textit{Pre-instruction}: Zero-knowledge proof verification
    $\big((\ct_\current, \ct_\new), \pirange)\big)$
  \item \textit{In-program}: The encrypted available balance in the
    \texttt{ZK-Token} account to be updated is equal to $\ct_\current$ in the
    instruction data.
  \item \textit{In-program}: The pending balance $\ct_\pending$ in the
    \texttt{ZK-Token} account to be updated is equal to a (deterministic)
    encryption of 0:
    \[ \ct_\pending = (0H + 0G, 0P) .\]
  \end{itemize}

\newpage
\subsubsection{$\ApplyPendingBalance$}
Apply the pending balance of a \texttt{ZK-Token} account to its available
balance.
\vspace{0.5em}

\noindent
\textit{Addresses to be specified}:
\begin{itemize}[leftmargin=2.5cm]
  \item[$\writable$] The \texttt{ZK-Token} account
  \item[] The corresponding \texttt{Token} account
  \item[$\signer$/$\none$] The single account owner or the multisig account owner
  \item[$\signer$, $\opt$] Required $M$ signer accounts for the \texttt{Token}
    Multisig account
\end{itemize}

\noindent
\textit{Data to be specified}:
\begin{itemize}
  \item[] (none)
\end{itemize}

\noindent
\textit{Cryptographic checks}:
\begin{itemize}
  \item[] (none)
\end{itemize}



\newpage
\subsubsection{$\DisableInboundTransfers$}
Disable incoming transfers to a \texttt{ZK-Token} account.
\vspace{0.5em}

\noindent
\textit{Addresses to be specified}:
\begin{itemize}[leftmargin=2.5cm]
  \item[$\writable$] The \texttt{ZK-Token} account
  \item[] The corresponding \texttt{Token} account
  \item[$\signer$/$\none$] The single account owner or the multisig account owner
  \item[$\signer$, $\opt$] Required $M$ signer accounts for the \texttt{Token}
    Multisig account
\end{itemize}

\noindent
\textit{Data to be specified}:
\begin{itemize}
  \item[] (none)
\end{itemize}

\noindent
\textit{Cryptographic checks}:
\begin{itemize}
  \item[] (none)
\end{itemize}


\newpage
\subsubsection{$\EnableInboundTransfers$}
Enable incoming transfers to a \texttt{ZK-Token} account.
\vspace{0.5em}

\noindent
\textit{Addresses to be specified}:
\begin{itemize}[leftmargin=2.5cm]
  \item[$\writable$] The \texttt{ZK-Token} account
  \item[] The corresponding \texttt{Token} account
  \item[$\signer$/$\none$] The single account owner or the multisig account owner
  \item[$\signer$, $\opt$] Required $M$ signer accounts for the \texttt{Token}
    Multisig account
\end{itemize}

\noindent
\textit{Data to be specified}:
\begin{itemize}
  \item[] (none)
\end{itemize}

\noindent
\textit{Cryptographic checks}:
\begin{itemize}
  \item[] (none)
\end{itemize}

\endgroup

\newcommand{\PedComm}{\ms{PedComm}}
\newcommand{\DecHandle}{\ms{DecHandle}}
\newcommand{\lo}{\ms{lo}}
\newcommand{\hi}{\ms{hi}}

\newpage
\subsection{Confidential Transfers}
The most cryptographically advanced component of the \texttt{ZK-Token} program
is the $\Transfer$ instruction. We refer to Section~\ref{sec:description} for the
high level intuition behind how transfers are done in the \texttt{ZK-Token}
program. Here, we provide some of the remaining implementation details for the
$\Transfer$ instruction.

\paragraph{Encrypting the transfer amount.} 
As described in
Section~\ref{sec:description}, a transfer amount $0 \le x \le 2^{64}$ is
encrypted as two 32-bit numbers $\xlo$ and $\xhi$ such that $x = \xlo +
2^{32} \cdot \xhi$. These two numbers must be encrypted under three public keys that are
associated with the source, destination, and auditor accounts for the transfer.
Recall from Section~\ref{sec:encryption} that an ElGamal ciphertext consists of
two components:
\begin{itemize}
  \item The \emph{Pedersen commitment} component that depends on the message to
    be encrypted, but is independent of the public key.
  \item The \emph{decryption handle} component that depends on the public key,
    but is independent of the message to be encrypted.
\end{itemize}
As the Pedersen commitment component is independent of public keys, it can be
shared among encryption of a single message under multiple public keys.
Therefore, the ElGamal ciphertext that encrypts a transfer amount consist of the
following components:
\begin{itemize}
  \item \emph{Message commitments}: 
    \begin{itemize}
      \item $\comm_{\xlo} = \PedComm(\xlo)$
      \item $\comm_{\xhi} = \PedComm(\xhi)$
    \end{itemize}

  \item \emph{Decryption handles for $\xlo$}: 
    \begin{itemize}
      \item $\handle_{\source, \lo} = \DecHandle_\lo(\pk_\source)$
      \item $\handle_{\dest, \lo} = \DecHandle_\lo(\pk_\dest)$
      \item $\handle_{\auditor, \lo} = \DecHandle_\lo(\pk_\auditor)$
    \end{itemize}

  \item \emph{Decryption handles for $\xhi$}: 
    \begin{itemize}
      \item $\handle_{\source, \hi} = \DecHandle_\hi(\pk_\source)$
      \item $\handle_{\dest, \hi} = \DecHandle_\hi(\pk_\dest)$
      \item $\handle_{\auditor, \hi} = \DecHandle_\hi(\pk_\auditor)$
    \end{itemize}

\end{itemize}

\newcommand{\final}{\ms{final}}

\paragraph{Zero-knowledge proofs.}
A transfer instruction must include zero-knowledge proofs that consist of two
parts:
\begin{itemize}
  \item \emph{Range proof}: This component should certify that the following
    Pedersen commitments are for the messages of the correct range:
    \begin{itemize}
      \item The commitment $\comm_{\final} = \comm_{\available} - (\comm_{\xlo} +
        2^{32} \cdot \comm_{\xhi})$ is a commitment of a 64-bit numbers where
        $\comm_{\available}$ is the Pedersen commitment associated with the
        source encrypted spendable balance.

      \item The commitment $\comm_{\xlo}$ is a commitment of a 32-bit number.

      \item The commitment $\comm_{\xhi}$ is a commitment of a 32-bit number.
    \end{itemize}
    As described in Section~\ref{sec:rangeproof}, instead of having three
    independent range proofs for the three commitments, one can generate a
    single aggregate range proof that certifies validity of the three
    commitments.

  \item \emph{Ciphertext validity proof}: This component should certify that the
    decryption handles are generated correctly for each public keys
    $\pk_\source$, $\pk_\dest$, and $\pk_\auditor$.

\end{itemize}

\paragraph{Two transfer instructions.}
A single transfer instruction must contain the encrypted transfer amount and the
zero-knowledge proofs certifying the validity of the ciphertexts. Currently, the
size of a Solana program instruction is capped at 1232 bytes. This prevents the
ciphertexts and the zero-knowledge proofs from fitting into a single transfer
instruction. Therefore, the transfer instruction is split into two separate
instructions, each containing independent ciphertext and zero-knowledge proof
components of a transfer.
\begin{itemize}
  \item The $\TransferWithRangeProof$ instruction contains the Pedersen
    commitments $\comm_{\xlo}$, $\comm_{\xhi}$ of the encrypted transfer amount
    and the associated range proof certifying these commitments.

  \item The $\TransferWithValidityProof$ instruction contains the decryption
    handles 
    \begin{itemize}
      \item $\handle_{\source, \lo}$, $\handle_{\dest, \lo}$, $\handle_{\auditor, \lo}$, 
      \item $\handle_{\source, \hi}$, $\handle_{\dest, \hi}$,
        $\handle_{\auditor, \hi}$,
    \end{itemize}
    and the associated validity proofs for these handles.
\end{itemize}

\noindent
A transfer of tokens between two \texttt{ZK-Token} accounts are fully executed
only when both of these transfer instructions are executed by the
\texttt{ZK-Token} program. 

\todo{Discuss ephemeral state}


\begingroup
\setlength{\parindent}{0em}
\setlength{\parskip}{0.75em}


\newpage
\subsubsection{$\TransferWithRangeProof$}

\ignore{
Transfers tokens confidentially from one account to another. 

Along with $\TransferWithValidityProof$, this instruction is one of two transfer
instructions that make up a full transfer. This instruction must include
Pedersen commitments of the transfer amount and the associated
rangeproof~\ref{sec:rangeproof}.

This instruction will fail if:
\begin{itemize}
  \item Either the source or the destination is frozen
  \item The destination has disabled incoming transfers by invoking
    $\DisableInboundTransfers$
\end{itemize}
\vspace{0.5em}

\noindent
\textit{Addresses to be specified}:
\begin{itemize}[leftmargin=2cm]
  \item[$\writable$] The source \texttt{ZK-Token} account
  \item[] The corresponding \texttt{Token} account for the source
    \texttt{ZK-Token} account
  \item[$\writable$] The destination \texttt{ZK-Token} account
  \item[] The corresponding \texttt{Token} account for the destination
    \texttt{ZK-Token} account
  \item[$\signer$/$\none$] The single source account owner or the multisig
    source account owner
  \item[$\signer$, $\opt$] Required $M$ signer accounts for the \texttt{Token} Multisig
    account
  \item[] The transfer auditor account
\end{itemize}

\noindent
\textit{Data to be specified}:
\begin{itemize}
  \item $(\comm_\lo, \comm_\hi)$: Pedersen commitments of the 64-bit
    transfer amount divided into two 32-bit values

  \item $\comm_\source$: The expected Pedersen commitment of the source
    \texttt{ZK-Token} account available balance.

    This component can be computed by the program itself; it is included as part
    of the instruction so that the range proof $\pirange$ can be verified as a
    pre-instruction. \todo{This is contained in ephemeral state.}

  \item \todo{ephemeral state}

  \item $\pirange$: Zero-knowledge proof certifying that $\comm_\final$ is a
    commitment to a 64-bit value and $\comm_\lo, \comm_\hi$ are commitments to
    32-bit values.

\end{itemize}

\noindent
\textit{Cryptographic checks}:
\begin{itemize}
  \item \textit{Pre-instruction}: Zero-knowledge proof verification
    $(\comm_\final, \comm_\lo, \comm_\hi, \pirange)$
  \item \textit{In-program}: Let $\comm_\available$ be the encrypted available
    balance in the source \texttt{ZK-Token} account. The following relation must
    be verified:
    \[ \comm_\final = \comm_\available - (\comm_\lo + 2^{32} \cdot \comm_\hi) .\]
  \item \textit{In-program}: \todo{ephemeral state}
\end{itemize}

}

\newpage
\subsubsection{$\TransferWithValidityProof$}

\ignore{
\todo{Need to update this instruction...}
Transfers tokens confidentially from one account to another. 

Along with $\TransferWithRangeProof$, this instruction is one of two transfer
instructions that make up a full transfer. This instruction must include
decryption handles for the encryption keys associated with the source,
destination (see Section~\ref{sec:elgamal-description}) and auditor accounts as
well as proofs that these decryption handles are well-formed (See
Section~\ref{sec:rangeproof}). 

This instruction will fail if:
\begin{itemize}
  \item Either the source or the destination is frozen
  \item The destination has disabled incoming transfers by invoking
    $\DisableInboundTransfers$
\end{itemize}
\vspace{0.5em}

\noindent
\textit{Addresses to be specified}:
\begin{itemize}[leftmargin=2cm]
  \item[$\writable$] The source ZK-Token account
  \item[] The corresponding SPL Token account for the source ZK-Token account
  \item[$\writable$] The destination ZK-Token account
  \item[] The corresponding SPL Token account for the destination ZK-Token
    account
  \item[$\signer$/$\none$] The single source account owner or the multisig
    source account owner
  \item[$\signer$, $\opt$] Required $M$ signer accounts for the SPL Token Multisig
    account
  \item[] The transfer auditor account
\end{itemize}

\noindent
\textit{Data to be specified}:
\begin{itemize}
  \item \todo{Need to update this part...}

  \item $(\handle_\lo, \handle_\hi)$: The decryption handle components of the
    ElGamal encryption of the 64-bit transfer amount divided into 32-bit values.

  \item $\handle_\source$: The expected decryption handle of the source ZK-Token
    account available balance.

  \item $\pivalidity$: Zero-knowledge proof certifying that the decryption
    handles $\handle_\lo, \handle_\hi$, and $\handle_\source$ are well-formed.

\end{itemize}

\noindent
\textit{Pre-instruction verification}:
\begin{enumerate}[leftmargin=1.25cm]
  \item Instruction signatures

  \item Validity proof $\big((\handle_\lo, \handle_\hi, \handle_\source),
    \pivalidity\big)$
\end{enumerate}

\noindent
\textit{In-program verification}
\begin{enumerate}[leftmargin=1.25cm]
  \item The encrypted available balance in the confidential account to be
    updated is equal to $\ct_\current$ in the instruction data.
\end{enumerate}


\newpage

\endgroup

}
