\section{Construction}

To be written...

\newcommand{\RecursiveHash}{\ms{RecursiveHash}}
\newcommand{\croot}{c_{\ms{root}}}
\newcommand{\cleft}{c_{\ms{left}}}
\newcommand{\cright}{c_{\ms{right}}}
\newcommand{\midi}{\ms{mid}}
\newcommand{\RecursiveProve}{\ms{RecursiveProve}}
\newcommand{\tagleft}{\ms{L}}
\newcommand{\tagright}{\ms{R}}

\begin{construction}
  \label{cons:main}
  Let $\calD = \calD_{\lambda \in \N}$ be a domain space of inputs, which
  includes the set of positive integers $\N \subseteq \calD$. Let $\calH_\calK =
  \{ H_k: \calD \times \calD \to \calD\}_{k \in \calK}$ for a key space
  $\calK_{\lambda \in \N}$. We construct an accumulator scheme $\Piacc =
  (\Commit, \Prove, \Verify)$ with domain $\calX$, commitment space $\calY$, and
  proof space $\calD^*$ as follows:
  \begin{itemize}
    \item $\Commit(1^\lambda, x_0, \ldots, x_{\ell-1}) \to c$: On input the
      security parameter $\lambda$, sequence of inputs $x_0, \ldots,
      x_{\ell-1}$, the commit algorithm hashes the sequence of inputs using
      the recursive algorithm $\croot \gets \RecursiveHash(x_0, \ldots,
      x_{\ell-1})$ that is defined as follows:
      \begin{description}
        \item $\RecursiveHash(x_0, \ldots, x_{\ell-1})$:
          \begin{itemize}
            \item If $\ell = 1$, then simply return $x_0$.
            \item Otherwise, let $\midi \gets \lceil (\ell-1)/2 \rceil$. The
              function recursively computes $\cleft \gets \RecursiveHash(x_0,
              \ldots, x_\midi)$, $\cright \gets \RecursiveHash(x_{\midi + 1},
              \allowbreak x_{\ell-1})$, and then returns $H(\allowbreak \cleft,
              \allowbreak \cright)$.
          \end{itemize}
      \end{description}
      The commit algorithm finally computes $c \gets H(\croot, \ell)$ and return
      $c$ as the commitment.

    \item $\Prove(1^\lambda, x_1, \ldots, x_\ell, i) \to \pi$: On input the security
      parameter $\lambda$, set of inputs $x_0, \ldots, x_{\ell-1}$, and an index
      $i \in [\ell]$, the proving algorithm proceeds as follows. It maintains a
      set $S \subset \calD \times \{\tagleft, \tagright\}$ of pairs of domain
      elements in $\calD$ and tag $\{\tagleft, \tagright\}$ that is initially
      empty $S = \{ \}$. The algorithm adds elements to the set $S$ using the
      recursive algorithm $\RecursiveProve(S, x_0, \ldots, x_{\ell-1}, i)$ as
      follows:
      \begin{description}
        \item $\RecursiveProve(S, x_0, \ldots, x_{\ell-1}, i)$:
          \begin{itemize}
            \item If $\ell = 1$, then simply return.
            \item Otherwise, let $\midi \gets \lceil (\ell-1)/2 \rceil$
              \begin{itemize}
                \item If $i \le \midi$, then compute $\cright \gets
                  \RecursiveHash(x_{\midi + 1}, \ldots, x_{\ell-1})$ and add $S
                  \gets S \cup \{(\cright, \tagright)\}$. Then recursively compute
                  $\RecursiveProve(S, x_0, \ldots, x_\midi, i)$.
                \item If $i > \midi$, then compute $\cleft \gets
                  \RecursiveHash(x_0, \ldots, x_\midi)$ and add $S \gets S \cup
                  \{(\cleft, \tagleft)\}$. Then recursively compute
                  $\RecursiveProve(S, x_{\midi+1}, \ldots, x_{\ell-1}, i -
                  \midi)$.
              \end{itemize}
          \end{itemize}
          The proving algorithm sets $\pi \gets (S, \ell)$ and return $\pi$ as
          the proof.
      \end{description}

    \item $\Verify(c, x, \pi) \to b$: On input a commitment $c \in \calD$, input
      $x \in \calD$, and proof $\pi = (\ell, \pi)$, the verification algorithm
      first parses the proof $\pi = \{ (c_1, \tau_1), \ldots, (c_t, \tau_t) \}$
      for some $t \in \N$. Then, for $i = 1, \ldots, t$, it iteratively computes
      the hash as follows:
      \begin{itemize}
        \item Let $h_0 \gets x$.
        \item For $(c_1, \tau_1), \ldots, (c_t, \tau_t)$, sequentially compute
          in order:
          \begin{itemize}
            \item If $\tau_i = \tagleft$, set $h_i
              \gets H(c_i, h_{i-1})$.
            \item If $\tau_i = \tagright$, set $h_i \gets H(h_{i-1}, c_i)$.
          \end{itemize}
      \end{itemize}
      Finally, the verification algorithm computes and verifies that $H(h_t,
      \ell) = c$. If the equality holds, the verification algorithm accepts
      (returns 1) and otherwise, it rejects (returns 0).

  \end{itemize}
\end{construction}

\begin{theorem}[Correctness]
    The accumulator scheme $\Piacc$ in Construction~\ref{cons:main} satisfies
    \emph{correctness} (Definition~\ref{def:correctness}).
\end{theorem}

\begin{theorem}[Compactness]
    The accumulator scheme $\Piacc$ in Construction~\ref{cons:main} satisfies
    \emph{compactness} (Definition~\ref{def:compactness}).
\end{theorem}

\begin{theorem}[Soundness]
    To be written...
\end{theorem}

